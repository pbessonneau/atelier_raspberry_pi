\section{Formatage et écriture de la carte SD}

\subsection{Téléchargement de l'image}

Les différentes images proposées par la Raspberry Pi Foundation sont à cette adresse \href{https://www.raspberrypi.org/downloads/}{https://www.raspberrypi.org/downloads/}.

Différents systèmes d'exploitation sont disponibles~:
\begin{itemize}
	\item NOOBS
	\item Raspbian
	\item Ubuntu Mate
	\item Snappy Ubuntu Core
	\item Windows 10
	\item OSMC
	\item LibreElec
	\item PINET
	\item RISC OS
	\item Weather Station
\end{itemize}

La NOOBS est un particulière.  NOOBS signifie \emph{New Out The Box Software} et a été développée spécifiquement pour le Pi par la Rapsberry Pi Foundation. NOOBS n'est pas vraiment un système d'exploitation mais un gestionnaire de systèmes qui va simplifier l'installation de l'OS désiré mais également permettre le multiboot entre plusieurs systèmes présents sur la même carte SD. Solution idéale pour débuter ou pour tester les différents systèmes existants.

Les distributions OSMC et LibreELEC utilisent les bonnes capacités vidéos des Pi pour en faire des serveurs multimedia. Pour les audiophiles il existe des amplificateurs ou des DACs (Digital-to-Analogic Converter) de grande qualité pour le prix de 35 à 45\$.

Les images sont disponibles sous forme de fichiers compressés au format zip. Pour télécharger l'image :
\begin{verbatim}
wget https://downloads.raspberrypi.org/raspbian_lite_latest
\end{verbatim}

La première étape consiste à vérifier qu'il n'y a pas eu de problème lors du téléchargement en vérifiant la somme de contrôle (SHA-1).

Après téléchargement, il suffit de taper~:
\begin{verbatim}
$sha1sum 2016-11-25-raspbian-jessie-lite.zip 
6741a30d674d39246302a791f1b7b2b0c50ef9b7  2016-11-25-raspbian-jessie-lite.zip
\end{verbatim}

Il suffit de comparer alors le checksum du 	site web à celui affiché dans la console.

Pour décompresser l'image, il suffit de faire~: 

\begin{verbatim}
$unzip 2016-09-23-raspbian-jessie-lite.zip
\end{verbatim}

Pour écrire l'image, vous pouvez vous aider de l'\href{https://www.raspberrypi.org/documentation/installation/installing-images/linux.md}{aide pour Linux} et vous reporter à l'aide sur l'utilisation de ``dd'' ci-dessous.

Mais il ne faut pas faire d'erreur en spécifiant le disque. Si vous êtes frileux, il y a le logiciels \href{http://etcher.io/}{etcher}. Ce logiciel fonctionne aussi bien sur Linux que Windows et Mac OS.

Les images représentent les données brutes à écrire sur une carte SD. Le format \og~img~\fg est aussi valable pour faire une image disque, etc.

L'utilitaire pour graver cette image est \emph{dd} sous Linux.

Il y a trois arguments~:
\begin{description}
	\item[bs] ou block-size. C'est la taille des blocs que va écrire l'utilitaire. Plus la valeur est grande plus l'écriture est rapide. Mais une taille trop grande peut générer des erreurs d'écriture. Le mieux est de choisir 1M ou 4M respectivement 1 méga-octet ou 4 méga-octets.
	\item[if] \emph{inuput file}, c'est la source. Ici cela va être notre image mais ça peut être un disque comme on le verra plus tard.
	\item[of] \emph{output file}, c'est la destination. Il faut faire extrêmement attention à la destination. En effet si vous écrivez sur un disque vous devez le faire généralement en tant que root et donc si vous vous trompez de disque vous allez PERDRE toutes les données du disque. Il convient de vérifier que l'on ne se trompe pas de destination.
\end{description}

Pour repérer le disque à écrire, la première possibilité est d'insérer la carte et de taper juste après la commande \emph{dmesg}~:

\begin{verbatim}
[ 6515.152466] usb 2-3.4: new high-speed USB device number 9 using xhci_hcd
[ 6515.242745] usb 2-3.4: New USB device found, idVendor=058f, idProduct=6362
[ 6515.242748] usb 2-3.4: New USB device strings: Mfr=1, Product=2, SerialNumber=3
[ 6515.242750] usb 2-3.4: Product: Mass Storage Device
[ 6515.242751] usb 2-3.4: Manufacturer: Generic
[ 6515.242752] usb 2-3.4: SerialNumber: 058F312D81B
[ 6515.243201] usb-storage 2-3.4:1.0: USB Mass Storage device detected
[ 6515.244066] scsi host5: usb-storage 2-3.4:1.0
[ 6516.241274] scsi 5:0:0:0: Direct-Access     Generic  USB SD Reader    1.00 PQ: 0 ANSI: 0
[ 6516.241872] scsi 5:0:0:1: Direct-Access     Generic  USB CF Reader    1.01 PQ: 0 ANSI: 0
[ 6516.242416] scsi 5:0:0:2: Direct-Access     Generic  USB SM Reader    1.02 PQ: 0 ANSI: 0
[ 6516.242941] scsi 5:0:0:3: Direct-Access     Generic  USB MS Reader    1.03 PQ: 0 ANSI: 0
[ 6516.243594] sd 5:0:0:0: Attached scsi generic sg1 type 0
[ 6516.243889] sd 5:0:0:1: Attached scsi generic sg2 type 0
[ 6516.244143] sd 5:0:0:2: Attached scsi generic sg3 type 0
[ 6516.248541] sd 5:0:0:3: Attached scsi generic sg4 type 0
[ 6516.764148] sd 5:0:0:0: [sdb] 62333952 512-byte logical blocks: (31.9 GB/29.7 GiB)
[ 6516.765711] sd 5:0:0:0: [sdb] Write Protect is off
[ 6516.765713] sd 5:0:0:0: [sdb] Mode Sense: 03 00 00 00
[ 6516.767313] sd 5:0:0:0: [sdb] No Caching mode page found
[ 6516.767316] sd 5:0:0:0: [sdb] Assuming drive cache: write through
[ 6516.769552] sd 5:0:0:2: [sdd] Attached SCSI removable disk
[ 6516.772199] sd 5:0:0:1: [sdc] Attached SCSI removable disk
[ 6516.772727] sd 5:0:0:3: [sde] Attached SCSI removable disk
[ 6516.785521]  sdb: sdb1 sdb2 sdb3
[ 6516.791350] sd 5:0:0:0: [sdb] Attached SCSI removable disk
[ 6517.205895] EXT4-fs (sdb2): mounted filesystem with ordered data mode. Opts: (null)
[ 6517.224103] EXT4-fs (sdb3): mounted filesystem with ordered data mode. Opts: (null)
\end{verbatim}

Sur ce listing, on voit que la carte SD a comme nom de périphérique assigné \emph{/dev/sdb}.

Pour vérifier que c'est ça, il faut utiliser \emph{fdisk}~:
\begin{verbatim}
$sudo fdisk /dev/sdb
\end{verbatim}

Puis appuyer sur \emph{p} (pour print, m pour l'aide).

Vous allez avoir quelque chose comme ça~:
\begin{verbatim}
Disque /dev/sdb : 29,7 GiB, 31914983424 octets, 62333952 secteurs
Unités : sectors of 1 * 512 = 512 octets
Sector size (logical/physical): 512 bytes / 512 bytes
I/O size (minimum/optimal): 512 bytes / 512 bytes
Disklabel type: dos
Disk identifier: 0x99176dab

Périphérique Amorçage Start      Fin Secteurs  Size Id Type
/dev/sdb1              2048 62332927 62330880 29,7G  b W95 FAT32
\end{verbatim}

On voit que la carte fait environ 30Go avec une partition formatée en fat32, c'est typiquement une carte quand vous la sortez de la boite\dots 

Vous pouvez aussi utiliser \emph{blkid} qui est moins informatif ou bien, si la partition est montée \emph{df -h}.

\subsection{Sauvegarde de la carte SD}

Avec ``sdb'' comme identifiant de disque, il faut utiliser la commande suivante pour écrire l'image~:
\begin{verbatim}
$dd bs=4M if=2016-11-25-raspbian-jessie-lite.img of/dev/sdb
$sudo sync
\end{verbatim}

L'instruction ``sudo sync'' force le cache écriture à se vider pour pouvoir éjecter la carte en toute sécurité. Il faut pour retirer la carte attendre que l'ordinateur vous ``rende'' la main.

L'image une fois écrite ne remplit pas la carte... généralement. Sur la carte après écriture, il y a trois partitions~:
\begin{enumerate}
 \item une partition FAT32 (Windows/Linux) qui contient les fichiers de démarrage de la Pi
 \item une partition ext4 (Linux) qui contient tous les fichiers de la distribution Raspbian
 \item une zone vide non partitionnée et  non formatée
 \end{enumerate}
 
 Apparemment il y a un dispositif automatique pour étendre la partition ext4 avec l'espace disque non partitionnée sur les dernières versions de Raspbian. Donc si vous voulez faire une troisième partition comme un ``/home'', il faudra partitionner AVANT de démarrer le Pi avec la carte sur un ordinateur GNU/Linux. 
 
Après démarrage si l'espace non partitionnée subsiste vous pouvez utilisez ``raspi-config'' pour étendre la deuxième partition ou avec fdisk.
 
Dans le chapitre ``personnalisation'', est abordé le cas où on veut faire une partition ``home'' séparée. Ce partitionnement est à faire sur un ordinateur GNU/Linux après l'écriture de la carte.