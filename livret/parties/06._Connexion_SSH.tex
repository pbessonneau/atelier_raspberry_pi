\section{Connexion SSH à la Pi}

Pour se connecter à la Pi, il faut que la Pi et l'ordinateur soient connectés tous les deux sur ethernet et que la Pi soit accessible depuis l'ordinateur (pas de routeur ou de pare-feu entre les deux par exemple). 

\subsection{Se connecter à la Pi}

Pour se connecter il suffit de taper~:
\begin{verbatim}
$ssh 192.168.0.56 -l pi
$raspberry (c'est le mot de passe par défaut)

The authenticity of host '192.168.0.56 (192.168.0.56)' can't be established.
ECDSA key fingerprint is SHA256:jxN8A+IwAD+axlznp4wLME8Tpi36yCVW8duJmvA0yfs.
Are you sure you want to continue connecting (yes/no)? 

$yes

Warning: Permanently added '192.168.0.56' (ECDSA) to the list of known hosts.
\end{verbatim}

Les Pis lors du premier démarrage créent une clef unique pour le serveur, il n'y a donc pas besoin de régénérer les clefs pour le serveur SSH.

\subsection{Créer une clef pour la connexion sans mot de passe}

Vous pouvez le faire sur votre poste linux, utiliser putty-keygen ou le faire dans la pi.

Pour générer un clef si vous n'en avez pas déjà, il faut utiliser ``ssh-keygen''

\begin{verbatim}
$ssh-keygen
Generating public/private rsa key pair.
Enter file in which to save the key (/home/pi/.ssh/id_rsa): 
$ (entrée) par défaut
Created directory '/home/pi/.ssh'.
Enter passphrase (empty for no passphrase): 
$ (taper un mot de passe, vide pour pas de mot de passe)
Enter same passphrase again: 
$ (confirmer le mot de passe)
Your identification has been saved in /home/pi/.ssh/id_rsa.
Your public key has been saved in /home/pi/.ssh/id_rsa.pub.
The key fingerprint is:
31:96:37:e1:0d:93:95:a2:51:ff:80:1a:79:be:bb:ad pi@raspberrypi
The key's randomart image is:
+---[RSA 2048]----+
|         .=o..   |
|        .+oB.    |
|        Bo*.+    |
|       ..O . o   |
|        S .   .  |
|           .     |
|          .      |
|           o     |
|          Eo.    |
+-----------------+
\end{verbatim}

Pour activer la connexion par clef, il suffit de quelques modifications~:
\begin{verbatim}
$mv .ssh/id_rsa.pub .ssh/authorized_keys
$vi .ssh/id_rsa
(copier la clef privée dans un fichier ``clef_raspberry'' du répertoire .ssh
de votre ordinateur)
$rm .ssh/id_rsa
\end{verbatim}

Renommer le fichier ``id\_rsa.pub'' en ``authorized\_keys'' permet notamment au serveur de reconnaître que l'ordinateur avec la clef privé correspondante (id\_rsa) est autorisée à se connecter  sur la Pi avec ce compte.

Il faut donc bien copier ``id\_rsa'' sous ce nom ou un autre dans votre compte personnel et dans le répertoire .ssh pour un ordinateur GNU/Linux ou Mac OS.

Avec putty sous Windows vous trouverez de l'aide à cette \href{https://www.howtoforge.com/ssh_key_based_logins_putty}{page}.  

Sous GNU/Linux (et Mac OS), après il suffit de configurer ou de créer le fichier ``.ssh/config'~:

\begin{verbatim}
Host monpi
    Hostname 192.169.0.56
    User pi
    Port 22
    IdentityFile clef_raspberry
\end{verbatim}

Vous pourrez vous connecter sans mot de passe simplement en tapant ``ssh monpi''.
