
% Pour ins�rer des dessins de Linux
\newcommand{\LinuxA}{\includegraphics[height=0.5cm]{Graphiques/linux.png}}
\newcommand{\LinuxB}{\includegraphics[height=0.5cm]{Graphiques/linux.png}\xspace}

% Macro pour les petits dessins pour les diff�rents OS.
\newcommand{\Windows}{\emph{Windows}\xspace}
\newcommand{\Mac}{\emph{Mac OS X}\xspace}
\newcommand{\Linux}{\emph{Linux}\xspace}
\newcommand{\MikTeX}{MiK\tex\xspace}

% Des raccourcis pour les commandes \LaTeX, \TeX, ...
\newcommand{\latex}{\LaTeX\xspace}
\newcommand{\latexe}{\LaTeXe\xspace}
\newcommand{\tex}{\TeX\xspace}

% Commande pour le mode Verbatim
\newcommand{\code}{\vspace{0.2cm}\begin{Verbatim}[frame=single,label=Code,fontsize=\small]}
\newcommand{\tinycode}{\vspace{0.2cm}\begin{Verbatim}[frame=single,label=Code,fontsize=\tiny]}

% From Framabook (www.framasoft.net)
\newcommand{\latexcom}[1]{{\mdseries\ttfamily\upshape\symbol{92}#1}}
\newcommand{\indexcom}[1]{%
  \index{#1@\protect\texttt{\symbol{92}#1}}}
\newcommand{\ltxcom}[1]{%
  \latexcom{#1}\indexcom{#1}}  

  \newcommand{\yslant}{0.5}
\newcommand{\xslant}{-0.6}

% knitr
\usepackage[]{graphicx}\usepackage[]{color}
%% maxwidth is the original width if it is less than linewidth
%% otherwise use linewidth (to make sure the graphics do not exceed the margin)
\makeatletter
\def\maxwidth{ %
  \ifdim\Gin@nat@width>\linewidth
    \linewidth
  \else
    \Gin@nat@width
  \fi
}
\makeatother

\definecolor{fgcolor}{rgb}{0.345, 0.345, 0.345}
\newcommand{\hlnum}[1]{\textcolor[rgb]{0.686,0.059,0.569}{#1}}%
\newcommand{\hlstr}[1]{\textcolor[rgb]{0.192,0.494,0.8}{#1}}%
\newcommand{\hlcom}[1]{\textcolor[rgb]{0.678,0.584,0.686}{\textit{#1}}}%
\newcommand{\hlopt}[1]{\textcolor[rgb]{0,0,0}{#1}}%
\newcommand{\hlstd}[1]{\textcolor[rgb]{0.345,0.345,0.345}{#1}}%
\newcommand{\hlkwa}[1]{\textcolor[rgb]{0.161,0.373,0.58}{\textbf{#1}}}%
\newcommand{\hlkwb}[1]{\textcolor[rgb]{0.69,0.353,0.396}{#1}}%
\newcommand{\hlkwc}[1]{\textcolor[rgb]{0.333,0.667,0.333}{#1}}%
\newcommand{\hlkwd}[1]{\textcolor[rgb]{0.737,0.353,0.396}{\textbf{#1}}}%

\usepackage{framed}
\makeatletter
\newenvironment{kframe}{%
 \def\at@end@of@kframe{}%
 \ifinner\ifhmode%
  \def\at@end@of@kframe{\end{minipage}}%
  \begin{minipage}{\columnwidth}%
 \fi\fi%
 \def\FrameCommand##1{\hskip\@totalleftmargin \hskip-\fboxsep
 \colorbox{shadecolor}{##1}\hskip-\fboxsep
     % There is no \\@totalrightmargin, so:
     \hskip-\linewidth \hskip-\@totalleftmargin \hskip\columnwidth}%
 \MakeFramed {\advance\hsize-\width
   \@totalleftmargin\z@ \linewidth\hsize
   \@setminipage}}%
 {\par\unskip\endMakeFramed%
 \at@end@of@kframe}
\makeatother

\definecolor{shadecolor}{rgb}{.97, .97, .97}
\definecolor{messagecolor}{rgb}{0, 0, 0}
\definecolor{warningcolor}{rgb}{1, 0, 1}
\definecolor{errorcolor}{rgb}{1, 0, 0}
\newenvironment{knitrout}{}{} % an empty environment to be redefined in TeX


% Commandes vari�es
\newcommand{\df}{\emph{data.frame}\xspace}
\newcommand{\liste}{\emph{list}\xspace}
\newcommand{\cad}{c'est-�-dire\xspace}
\newcommand{\Rcode}{code \includegraphics[height=0.75em,width=1em]{logo_R}}
\newcommand{\Sweave}{\emph{Sweave}\xspace}
\newcommand{\knitr}{\emph{knitr}\xspace}
