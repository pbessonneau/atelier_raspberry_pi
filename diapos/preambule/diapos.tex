% \documentclass{beamer}
%\usetheme{Warsaw}

% Pour les documents en francais...
	\usepackage[utf8]{inputenc}
	\usepackage[french]{babel}
	\usepackage[french]{varioref}

% Math?matiques
	\usepackage{amsmath}

% Caracteres speciaux suppl?mentaires
	\usepackage{latexsym,amsfonts}

% A documenter
	\usepackage{moreverb}

% Macros pour les paquets
	\usepackage{array}  			% N?cessaires pour les tableaux de la macro Excel.

% Outil suppl?mentaire pour les tableaux
	\usepackage{multirow}
	\usepackage{booktabs}
	\usepackage{xcolor} % alternating row colors in table, incompatible avec certains modules
	\usepackage{longtable}
	\usepackage{colortbl}

% Pour ins?rer des graphiques
	\usepackage{graphicx} 			% Graphique simples
	\usepackage{subfigure}			% Graphiques multiples

% Pour ins?rer des couleurs
	\usepackage{color}

% Rotation des objets et des pages
%	\usepackage{rotating}
%	\usepackage{lscape}

% Pour insrer du code source, LaTeX ou SAS par exemple.
	\usepackage{verbatim}
        \usepackage{moreverb}
	\usepackage{listings}
	\usepackage{fancyvrb}

%	\lstset{language=SAS,numbers=left}		% Par dfaut le listing est en SAS

% Pour ins?rer des hyperliens
  \usepackage{hyperref}

% American Psychological Association (for bibliographic references).
	\usepackage{apacite}

% Pour l'utilisation des macros
	\usepackage{xspace}

% Pour l'utilisation de notes en fin de document.
%	\usepackage{endnotes}

% Array
%	\usepackage{multirow}
%	\usepackage{booktabs}

% Rotation
%	\usepackage{rotating}

% En t?tes et pieds de pages
%	\usepackage{fancyhdr}
%	\usepackage{lastpage}


% Page layout

% By LaTeX commands
%\setlength{\oddsidemargin}{0cm}
%\setlength{\textwidth}{16cm}
%\setlength{\textheight}{24cm}
%\setlength{\topmargin}{-1cm}
%\setlength{\marginparsep}{0.2cm}

% fancyheader parameters
%\pagestyle{fancy}

%\fancyfoot[L]{{\small Formation \LaTeX, DEPP}}
%\fancyfoot[c]{}
%\fancyfoot[R]{{\small \thepage/\pageref{LastPage}}}

%\fancyhead[L]{}
%\fancyhead[c]{}
%\fancyhead[R]{}

% Pour ins?rer des dessins de Linux
\newcommand{\LinuxA}{\includegraphics[height=0.5cm]{Graphiques/linux.png}}
\newcommand{\LinuxB}{\includegraphics[height=0.5cm]{Graphiques/linux.png}\xspace}

% Macro pour les petits dessins pour les diff?rents OS.
\newcommand{\Windows}{\emph{Windows}\xspace}
\newcommand{\Mac}{\emph{Mac OS X}\xspace}
\newcommand{\Linux}{\emph{Linux}\xspace}
\newcommand{\MikTeX}{MiK\tex\xspace}
\newcommand{\latex}{\LaTeX\xspace}


\newcommand{\df}{\emph{data.frame}\xspace}
\newcommand{\liste}{\emph{list}\xspace}
\newcommand{\cad}{c'est-à-dire\xspace}

\title{Atelier StarinuX Raspberry Pi}
\author{Pascal Bessonneau}
%\institute{DEPP}
\date{12/2016}
%\subtitle{RStudio}


\newcommand{\hreff}[2]{\underline{\href{#1}{#2}\xspace}}
