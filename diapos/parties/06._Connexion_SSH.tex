\section{Connexion SSH à la Pi}

\subsection{Se connecter à la Pi}

\begin{frame}[containsverbatim]
\frametitle{Première connexion}

\begin{verbatim}
$ssh 192.168.0.56 -l pi
$raspberry (c'est le mot de passe par défaut)
The authenticity of host '192.168.0.56 (192.168.0.56)' can't be established.
ECDSA key fingerprint is SHA256:jxN8A+IwAD+axlznp4wLME8Tpi36yCVW8duJmvA0yfs.
Are you sure you want to continue connecting (yes/no)? 
$yes
Warning: Permanently added '192.168.0.56' (ECDSA) to the list of known hosts.
\end{verbatim}
\end{frame}

\subsection{Créer une clef pour la connexion sans mot de passe}

\begin{frame}[containsverbatim]
\frametitle{Connexion par clef}

Vous pouvez le faire sur votre poste linux, utiliser putty-keygen ou le faire dans la pi.

Pour générer un clef si vous n'en avez pas déjà, il faut utiliser ``ssh-keygen''
\end{frame}

\begin{frame}[containsverbatim]
\frametitle{Connexion par clef}

Pour activer la connexion par clef, il suffit de quelques modifications~:
\begin{verbatim}
$mv .ssh/id_rsa.pub .ssh/authorized_keys
$vi .ssh/id_rsa
(copier la clef privée dans un 
fichier ``clef_raspberry'' du répertoire .ssh)
$rm .ssh/id_rsa
\end{verbatim}
\end{frame}

\begin{frame}[containsverbatim]
\frametitle{Connexion par clef}

Après il suffit de après de configurer ou de créer le fichier ``.ssh/config'~:

\begin{verbatim}
Host monpi
    Hostname 192.169.0.56
    User pi
    Port 22
    IdentityFile clef_raspberry
\end{verbatim}

Vous pourrez vous connecter sans mot de passe simplement en tapant ``ssh monpi''.

\end{frame}


