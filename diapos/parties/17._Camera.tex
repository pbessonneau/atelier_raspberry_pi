\section{Application avec la caméra}

\subsection{Description des caméras}

\begin{frame}[containsverbatim]
\frametitle{Description}
Les caméras Pi sont de deux générations, les premières ont un capteur de 3 millions de pixels et les secondes de 5 millions.

Attention quand vous manipulez ces caméras, elle sont très sensibles à l'électricité statique. Par conséquent avant de les prendre en main il faut toucher un objet à la terre comme la cage de votre ordinateur ou un radiateur.

Elles se branchent sur le port CSI, entre le port HDMI et le jack audio/video. Les parties conductrices du ruban doivent se trouver vers la prise HDMI.

\end{frame}

\begin{frame}[containsverbatim]
\frametitle{Description}

Il y a deux types de caméras~:
\begin{itemize}
	\item la caméra standard. C'est une caméra qui prend des photos ou vidéos que dans le visible.
	\item la caméra IR. C'est une caméra qui peut capturer des infra-rouges. Attention il faut une source de lumière adaptée comme des LEDs infra-rouges ou un dispositif pour les caméras extérieures. Quand la lumière dans le visible est forte elle capture le ``visible'' comme la caméra standard. Elle a donc un double usage.
\end{itemize}

Pour activer la caméra il faut lancer ``raspi-config'' et dans le menu activer la caméra.

\end{frame}

\subsection{Capturer des images}

\begin{frame}[containsverbatim]
\frametitle{Capturer des images}

Le programme pour capturer des images s'appelle raspistill. 

pour capturer une image il suffit de faire~:

\begin{verbatim}
raspistill -o testcapture.jpg
\end{verbatim}

\end{frame}

\begin{frame}[containsverbatim]
\frametitle{Capturer des images}

Pour régler la définition, il y a les arguments ``w'' et ``h''.

\begin{verbatim}
raspistill -w 1920 -h 1080 -o fullhdcapture.jpg
\end{verbatim}

\end{frame}

\begin{frame}[containsverbatim]
\frametitle{Capturer des images}

Attention il y a un temps de latence avant la prise de vue ! Par défaut ce temps est de 5 secondes.

Pour éliminer ce temps de latence il faut utiliser le paramètre ``t''. Le temps à indiquer est un entier en millisecondes.

\begin{verbatim}
# pas de temps de latence
raspistill -t 1 -o tensecondcapture.jpg
# temps de latence d'une minute
raspistill -t 60000 -o tensecondcapture.jpg
\end{verbatim}

Le voyant de la caméra est rouge quand on prends un cliché.

\end{frame}


\subsection{Capturer des vidéos}

\begin{frame}[containsverbatim]
\frametitle{Capturer des vidéos}

Cette fois la commande est ``raspivid''. Par défaut elle encode en h264, un format propriétaire.

\begin{verbatim}
raspivid -o testvideo.h264
\end{verbatim}

Le temps de capture par défaut est de 5 secondes. 

\end{frame}

\begin{frame}[containsverbatim]
\frametitle{Capturer des vidéos}

Pour le modifier, il suffit d'utiliser l'argument ``t'' toujours en millisecondes

\begin{verbatim}
raspivid -t 60000 -o testvideo.h264
\end{verbatim}

\end{frame}

\begin{frame}[containsverbatim]
\frametitle{Capturer des vidéos}

Pour changer la résolution ce sont les mêmes arguments que pour ``raspistill''~:

\begin{verbatim}
raspivid -w 1280 -h 720 -t 60000 -o testvideo.h264
\end{verbatim}

Selon les modèles de caméras, les possibilités en matière de définition varient.

\end{frame}

\subsection{Timelapse video}

\begin{frame}[containsverbatim]
\frametitle{Timelapse}

Vous pouvez programmer la caméra pour prendre des photos à intervalle régulier. Par exemple pour prendre durant une minute une image toutes les secondes~:
\begin{verbatim}
raspistill -o frame%08d.jpg -tl 10000 -t 600000
\end{verbatim}

Dans ce cas vous verrez des fichiers appelés frame00000001.jpg, frame00000002.jpg, etc. dans le répertoire courant.

\end{frame}

\begin{frame}[containsverbatim]
\frametitle{Timelapse}

Pour les transformer en vidéo, il faut utiliser un utilitaire ``avconv''. Pour l'installer~:

\begin{verbatim}
$sudo apt-get install libav-tools
\end{verbatim}

\end{frame}

\begin{frame}[containsverbatim]
\frametitle{Timelapse}

Ensuite~:

\begin{verbatim}
avconv -r 10 -i frame%08d.jpg -r 10 -vcodec libx264 timelapse.mp4
\end{verbatim}

L'argument r indique le nombre de frames par secondes. L'argument vcodec indique le type de compression.

``avconv'' est un outil très complexe avec lequel on peut beaucoup de choses.

\end{frame}