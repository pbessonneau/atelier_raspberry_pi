\section{Sécurisation de votre Raspberry Pi}

\subsection{Changer le mot de passe}

\begin{frame}[containsverbatim]
\frametitle{Changement de mot de passe}

La première étape est de changer le mot de passe par défaut. Il suffit de se connecter et de faire~:
\begin{verbatim}
passwd
\end{verbatim}

Choisissez un mot de passe le plus sécuritaire possible. 
\end{frame}

\begin{frame}[containsverbatim]
\frametitle{Ajouter un utilisateur}

Vous pouvez aussi créer un utilisateur spécifique. Par exemple pour ajouter un utilisateur \emph{pascal}~:
\begin{verbatim}
adduser pascal
....
\end{verbatim}
\end{frame}


\begin{frame}[containsverbatim]
\frametitle{Pare-feu}

Il faut veiller si vous bricolez avec votre pi de lui donner les mêmes droits que l'utilisateur pi. Pour cela, il faut éditer le fichier \emph{/etc/groups}.

Ce sont surtout le groupe \emph{sudo}, \emph{SPI} qui sont utiles car ils permettront de lancer des commandes en tant que \emph{root} et d'avoir le contrôle sur les interfaces de la pi.
\begin{verbatim}

\end{verbatim}

\end{frame}

\subsection{Pare-feu}

\begin{frame}[containsverbatim]
\frametitle{Pare-feu}

Vous devez mettre un pare-feu. Surtout si vous utilisez le Pi comme serveur et donc qui est ``face'' à internet. 

\textcolor{red}{Attention}, quelque soit le pare-feu, il faut laisser le port 22 ouvert pour vous connecter avec SSH sinon votre Raspberry Pi deviendra une brique... Il faut ouvrir le port et lancer le pare-feu au cours d'une session SSH et se connecter dans une nouvelle connexion pour vérifier que le port est bien ouvert !

\end{frame}


\begin{frame}[containsverbatim]
\frametitle{Sauvegarde d'un pare-feu vide}

Utile si vous jouez avec iptables. Pour sauvegarder un pare-feu vide~:

\begin{verbatim}
$sudo iptables-save ~/iptables.empty
\end{verbatim}
\end{frame}


\begin{frame}[containsverbatim]
\frametitle{Sauvegarde d'un pare-feu vide}
Pour l'utiliser il faut faire~:

\begin{verbatim}
sudo iptables-restore ~/iptables.empty
\end{verbatim}
\end{frame}


\begin{frame}[containsverbatim]
\frametitle{Pare-feu à la main}

Vous pouvez faire un pare-feu simple ``à la main'' avec \emph{iptables}~:
\begin{verbatim}
#!/bin/sh

IPTABLES=/sbin/iptables

# Creation des tables
$IPTABLES -N TCP
$IPTABLES -N UDP
...
\end{verbatim}

\end{frame}

\begin{frame}[containsverbatim]
\frametitle{Pare-feu à la main}

Attention tous les fichiers du firewall doivent être accessibles seulement par le root.
\begin{verbatim}
$sudo chmod 700 02-firewall
$sudo chown root:root 02-firewall
\end{verbatim}

\end{frame}

\begin{frame}[containsverbatim]
\frametitle{Pare-feu à la main}

Le fichier du firewall est à mettre dans le répertoire ``/etc/network/if-pre-up.d''. Il sera exécuter dès que le réseau va devenir accessible. Vous pouvez aussi le placer en fin de lancement de Linux~: dans le fichier ``/etc/rc.local''.

\end{frame}

\begin{frame}[containsverbatim]
\frametitle{arno-iptables-firewall}

Sinon il y a un paquet très efficace et pratique~: \emph{arno-iptables-firewall}. C'est un script BASH qui fait un pare-feu de bonne facture. Il faut indiquer les ports à laisser ouvert lors de l'installation. Il gère aussi des règles plus complexes~: NAT, DMZ, ports ouverts sur le LAN et pas sur internet, ... 

Pour l'installer, c'est comme pour n'importe quel paquet Debian~:
\begin{verbatim}
$sudo apt-get install arno-iptables-firewall
\end{verbatim}

\end{frame}

\begin{frame}[containsverbatim]
\frametitle{arno-iptables-firewall}

Pour stopper le pare-feu, relancer et lancer le pare-feu les commandes sont respectivement~:
\begin{verbatim}
$sudo service arno-iptables-firewall stop
$sudo service arno-iptables-firewall restart
$sudo service arno-iptables-firewall start
\end{verbatim}

\end{frame}

\subsection{Mise à jour}

\begin{frame}[containsverbatim]
\frametitle{Mises à jour}

Il faut maintenir votre Raspberry Pi à jour... Pour tout ce qui est fourni par les paquets Debian, il suffit de faire de temps en temps un petit~:
\begin{verbatim}
$sudo apt-get update
$sudo apt-get upgrade -y
$sudo apt-get dist-upgrade -y
\end{verbatim}

Si la mise à jour contient une mise à jour du noyau, il faudra redémarrer le Pi.

\end{frame}

\begin{frame}[containsverbatim]
\frametitle{Mises à jour}

Comme vu précédemment il y a aussi le firmware à mettre à jour avec la commande spéciale~:
\begin{verbatim}
$sudo rpi-update
\end{verbatim}

\end{frame}

\begin{frame}[containsverbatim]
\frametitle{Mises à jour}

Le mieux est de créer un alias dans le fichier \emph{.bashrc}~:
\begin{verbatim}
alias update=sudo rpi-update && sudo apt-get update && \
sudo apt-get upgrade -y && sudo apt-get dist-upgrade -y 
\end{verbatim}

\end{frame}

\begin{frame}[containsverbatim]
\frametitle{Mises à jour}

Si vous avez installer un serveur web avec WordPress ou Yunohost il vous faudra aussi mettre à jour les composants en plus.

\end{frame}

\subsection{Bloquer les connexions SSH non basées sur les clefs}

\begin{frame}[containsverbatim]
\frametitle{Blocage de SSH par mot de passe}

Pour éviter les connexions qui utilisent un mot de passe au lieu de la clef privé/clef publique, il faut changer cette ligne (c'est la dernière ligne) dans ``/etc/ssh/sshd\_config''~:

\begin{verbatim}
UsePAM no
\end{verbatim}

\end{frame}

\begin{frame}[containsverbatim]
\frametitle{Blocage de SSH par mot de passe}

Il faut aussi empêcher les connexions en tant que root (c'est la ligne 28)~:
\begin{verbatim}
PermitRootLogin no
\end{verbatim}

Puis il faut redémarrer le serveur~:

\begin{verbatim}
$sudo service ssh restart
\end{verbatim}

\end{frame}
